\documentclass[10pt,twocolumn]{article}
\usepackage[utf8]{inputenc}
\usepackage[spanish]{babel}
\usepackage{lipsum}
\usepackage{hyperref}

\title{
  Especificación de Requerimientos de Software para el \\
  \textbf{Sistema de Gestión de Documentación Académica (Unify) de la UNRC}
}

\author{
  Tomás Rodeghiero\textsuperscript{1}, Nicolle Rosatti\textsuperscript{2}, Joaquín Tissera\textsuperscript{3} \\
  \textit{Departamento de Ciencias de la Computación} \\
  \textit{Facultad de Ciencias Exactas, Físico-Químicas y Naturales} \\\textit{Universidad Nacional de Río Cuarto}
}
\date{Año 2024}

\begin{document}

\maketitle

\begin{abstract}
Unify emerge como una solución integral para la gestión y difusión de documentos académicos de la Universidad Nacional de Río Cuarto (UNRC). Este sistema promueve la cooperación y el intercambio de conocimientos mediante una plataforma intuitiva y segura. Su finalidad es democratizar el acceso a recursos educativos y fomentar un ambiente de colaboración que contribuya al progreso académico de su comunidad.
\end{abstract}

\renewcommand{\thesection}{\Roman{section}}

\section{Introducción}
El presente documento detalla los requerimientos para el desarrollo del sistema Unify, destinado a revolucionar la gestión de contenidos académicos en la Universidad Nacional de Río Cuarto. Mediante una interfaz amigable y funcionalidades avanzadas, Unify facilitará el acceso a información relevante y promoverá la interacción enriquecedora entre los usuarios de la comunidad educativa. Más allá de ser un mero repositorio de documentos, Unify se concibe como un ecosistema digital que promueve la creación de conocimiento colectivo.

\section{Objetivos Generales}
Unify tiene como propósitos principales:
\begin{itemize}
  \item Ampliar el acceso a documentación académica de relevancia.
  \item Incentivar la colaboración productiva entre alumnos y docentes.
  \item Implementar un proceso eficaz para la evaluación y aprobación de contenidos.
  \item Desarrollar un sistema de reconocimientos que valore las contribuciones significativas a la comunidad educativa.
  \item Garantizar la pertinencia y excelencia del material disponible para el estudio.
\end{itemize}

\section{Descripción del Sistema}
Unify se presentará como una plataforma web en la que los usuarios, mediante la creación de perfiles personales, podrán subir y acceder a documentos académicos previamente aprobados. Estos recursos estarán organizados según carrera, asignatura y año académico correspondiente, y el sistema facilitará la búsqueda de información mediante filtros específicos por universidad, carrera y materia.

\subsection{Requerimientos de Usuarios}
Los usuarios de Unify podrán:
\begin{itemize}
  \item Registrarse y gestionar perfiles personales dentro de la plataforma.
  \item Contribuir con documentos sujetos a evaluación por un equipo de moderadores.
  \item Consultar y valorar los documentos aprobados.
  \item Realizar búsquedas específicas utilizando filtros avanzados.
  \item Participar en un sistema de recompensas por colaboración y calidad en las contribuciones.
\end{itemize}

\subsection{Requerimientos del Sistema}
Para satisfacer las necesidades de la comunidad, Unify contará con:
\begin{itemize}
  \item Un sistema de gestión de usuarios robusto, con diferentes niveles de roles y permisos.
  \item Una infraestructura de almacenamiento seguro y organizado para los documentos.
  \item Un algoritmo de búsqueda eficaz para facilitar la localización de recursos.
  \item Un proceso de revisión riguroso para asegurar la integridad académica y la calidad del contenido disponible.
\end{itemize}

\section{Requerimientos Funcionales}
Adicionalmente a las capacidades previamente mencionadas, Unify integrará:
\begin{itemize}
  \item Un sistema de autenticación que incluirá funciones de recuperación de contraseñas y gestión de seguridad de la cuenta.
  \item Una interfaz de entrenamiento interactivo apoyada por inteligencia artificial para facilitar la revisión de temas y la preparación de exámenes.
  \item Opciones de personalización de perfil para usuarios con categorías Premium en adelante, permitiendo una mayor expresión individual dentro de la comunidad.
  \item Un sistema de alertas y notificaciones para mantener a los usuarios informados sobre nuevas adiciones de documentos o actualizaciones importantes dentro de la plataforma.
\end{itemize}

\section{Requerimientos No Funcionales}
Para el diseño e implementación de Unify, se considerarán aspectos fundamentales como:
\begin{itemize}
  \item Seguridad de los Datos y Privacidad: Implementación de protocolos avanzados para la protección de la información personal y académica de los usuarios.
  \item Disponibilidad y Escalabilidad: Arquitectura del sistema preparada para adaptarse a un crecimiento significativo de la base de usuarios y al aumento de la carga de trabajo sin comprometer el rendimiento.
  \item Usabilidad y Accesibilidad: Diseño orientado al usuario, asegurando una experiencia intuitiva y accesible para todos los miembros de la comunidad universitaria, independientemente de su familiaridad con la tecnología.
  \item Rendimiento: Optimización de los tiempos de respuesta del sistema, garantizando búsquedas rápidas y un acceso eficiente a los documentos y recursos disponibles.
\end{itemize}

\newpage
\section{Consideraciones Finales}
El desarrollo de Unify representa un paso adelante en la modernización de los recursos académicos en la UNRC, con el potencial de impactar positivamente en la forma en que estudiantes y docentes interactúan y acceden a la información educativa. La implementación exitosa de este sistema requerirá una colaboración estrecha entre desarrolladores, usuarios finales y la administración universitaria para asegurar que se satisfagan todas las necesidades y expectativas planteadas en esta especificación de requerimientos.

\end{document}